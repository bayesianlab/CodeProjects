\documentclass[]{article}

\usepackage{mathtools}
\usepackage{amsfonts}
\usepackage{amssymb}
\usepackage{fullpage}
\usepackage{amsmath}
\usepackage{multirow}
\usepackage{graphicx}
\usepackage{caption}
\usepackage{subcaption}
\usepackage{float}
\usepackage{hyperref}
\hypersetup{
	colorlinks,
	citecolor=black,
	filecolor=black,
	linkcolor=black,
	urlcolor=black
}
\usepackage[framed,numbered,autolinebreaks,useliterate]{mcode}
\usepackage{listings}
\usepackage{color} %red, green, blue, yellow, cyan, magenta, black, white
\definecolor{mygreen}{RGB}{28,172,0} % color values Red, Green, Blue
\definecolor{mylilas}{RGB}{170,55,241}

\linespread{1.5}
%opening
\title{PS 3}
\author{Dillon Flannery-70923746}

\begin{document}

\maketitle
\section*{Problem 1}
$
X_t = \begin{pmatrix}
y_t\\ \pi_t \\ i_t \\ g_t \\ E_ty_{t+1} \\ E_t\pi_{t+1} \\	E_t g_{t+1} \\ \eta_t^4 \\ \eta_{t-1}^4 \\ \eta_{t-2}^4 \\ \eta_{t-3}^4 \\ \eta_{t}^3 \\ \eta_{t-1}^3 \\ \eta_{t-2}^3 \\ \eta_{t}^2 \\ \eta_{t-1}^2 \\ \eta_t^1
\end{pmatrix}
$
$
X_{t-1} = \begin{pmatrix}
y_{t-1}\\ \pi_{t-1}\\ i_{t-1} \\ g_{t-1} \\ E_{t-1}y_{t} \\ E_{t-1}\pi_{t} \\E_{t-1} g_t \\	\eta_{t-1}^4 \\ \eta_{t-2}^4 \\ \eta_{t-3}^4 \\ \eta_{t-4}^4 \\ \eta_{t-1}^3 \\ \eta_{t-2}^3 \\ \eta_{t-3}^3 \\ \eta_{t-1}^2 \\ \eta_{t-2}^2 \\ \eta_{t-1}^1
\end{pmatrix}
$ 

$ w_t  = 
\begin{pmatrix}
\epsilon_t^0	\\
\eta_t^1 \\
\eta_t^2\\
\eta_t^3\\
\eta_t^4 \\
\eta_t^5 \\
\end{pmatrix}
$ \\

$ \epsilon = 
\begin{pmatrix}
\epsilon_t^y \\
\epsilon_t^{\pi} \\
\epsilon_t^{g} \\
\end{pmatrix} $ \\
In state space form:
\[
\Gamma_0 X_t = \Gamma_1 X_{t-1} + \Psi w_t + \Pi \epsilon_t
\]
With the coefficients defined below. \\
$ \Gamma_0 $ = 
\begin{tabular}{r|rrrrrrrrrrrrrrrrr}
$ y_t $ & 1 & 0 & $ \sigma $ & 0 & -1 & 1 & 1 & 0 & 0 & 0 & 0 & 0 & 0 & 0 & 0 & 0 & 0 \\
$  \pi_t $ &$ -\kappa $ & 0 & 0 & 0 & 0 & $ -\beta $ & 1 & 0 & 0 & 0 & 0 & 0 & 0 & 0 & 0 & 0 & 0 \\
$ i_t $ & $ -\chi_y  $ & $ -\chi_{\pi} $ & 0 & 0 & 0 & 0 & 0 & 0 & 0 & 0 & 0 & 0 & 0 & 0 & 0 & 0 & 0 \\
$ g_t $ & 0 & 0 & 0 & 1 & 0 & 0 & 0 & 0 & 0 & 0 & 0 & 0 & 0 & 0 & 0 & 0 & 0 \\
$ E_ty_{t+1} $ & 1 & 0 & 0 & 0 & 0 & 0 & 0 & 0 & 0 & 0 & 0 & 0 & 0 & 0 & 0 & 0 & 0 \\
$ E_t\pi_{t+1}  $ & 0 & 1 & 0 & 0 & 0 & 0 & 0 & 0 & 0 & 0 & 0 & 0 & 0 & 0 & 0 & 0 & 0 \\
$ E_t g_{t+1} $ & 0 & 0 & 1 & 0 & 0 & 0 & 0 & 0 & 0 & 0 & 0 & 0 & 0 & 0 & 0 & 0 & 0 \\
$ \eta_t^4 $     & 0 & 0 & 0 & 0 & 0 & 0 & 0 & 1 & 0 & 0 & 0 & 0 & 0 & 0 & 0 & 0 & 0 \\
$ \eta_{t-1}^4 $ & 0 & 0 & 0 & 0 & 0 & 0 & 0 & 0 & 1 & 0 & 0 & 0 & 0 & 0 & 0 & 0 & 0 \\
$ \eta_{t-2}^4 $ & 0 & 0 & 0 & 0 & 0 & 0 & 0 & 0 & 0 & 1 & 0 & 0 & 0 & 0 & 0 & 0 & 0 \\
$ \eta_{t-3}^4 $ & 0 & 0 & 0 & 0 & 0 & 0 & 0 & 0 & 0 & 0 & 1 & 0 & 0 & 0 & 0 & 0 & 0 \\
$ \eta_{t}^3 $   & 0 & 0 & 0 & 0 & 0 & 0 & 0 & 0 & 0 & 0 & 0 & 1 & 0 & 0 & 0 & 0 & 0 \\
$ \eta_{t-1}^3 $ & 0 & 0 & 0 & 0 & 0 & 0 & 0 & 0 & 0 & 0 & 0 & 0 & 1 & 0 & 0 & 0 & 0 \\
$ \eta_{t-2}^3 $ & 0 & 0 & 0 & 0 & 0 & 0 & 0 & 0 & 0 & 0 & 0 & 0 & 0 & 1 & 0 & 0 & 0 \\
$ \eta_{t}^2 $   & 0 & 0 & 0 & 0 & 0 & 0 & 0 & 0 & 0 & 0 & 0 & 0 & 0 & 0 & 1 & 0 & 0 \\
$ \eta_{t-1}^2 $ & 0 & 0 & 0 & 0 & 0 & 0 & 0 & 0 & 0 & 0 & 0 & 0 & 0 & 0 & 0 & 1 & 0 \\
$ \eta_{t}^4 $   & 0 & 0 & 0 & 0 & 0 & 0 & 0 & 0 & 0 & 0 & 0 & 0 & 0 & 0 & 0 & 0 & 1 \\
\end{tabular}

$ \Gamma_1 $ = 
\begin{tabular}{r|rrrrrrrrrrrrrrrrr}
	$ y_{t-1} $ & 0 & 0 & 0 & 0 & $ \rho $ & 0 & 0 & 0 & 0 & 0 & 1 & 0 & 0 & 1 & 0 & 1 & 0 \\
	$  \pi_{t-1} $ &0  & $ \kappa\Gamma \rho $ & 0 & 0 & 0 &  0 & 0 & 0 & 0 & 0 & $ -\kappa\Gamma $ & 0 & 0 &  $ -\kappa\Gamma $ & 0 &  $ -\kappa\Gamma $ & 0 \\
	$ i_{t-1} $ & 0 & 0 & 0 & $ -\chi_y\Gamma\rho $ & 0 & 0 & 0 & 0 & 0 & 0 & $ -\chi_y\Gamma $ & 0 & 0 & $ -\chi_y\Gamma $ & 0 & $ -\chi_y\Gamma $ & 0 \\
	$ g_{t-1} $ & 0 & 0 & 0 & $ \rho $ & 0 & 0 & 0 & 1 & 0 & 0 & 1 & 0 & 0 & 1 & 0 & 1 & 0 \\
	$ E_{t-1}y_{t} $    & 0 & 0 & 0 & 0 & 1 & 0 & 0 & 0 & 0 & 0 & 0 & 0 & 0 & 0 & 0 & 0 & 0 \\
	$ E_{t-1}\pi_{t}  $ & 0 & 0 & 0 & 0 & 0 & 1 & 0 & 0 & 0 & 0 & 0 & 0 & 0 & 0 & 0 & 0 & 0 \\
	$ E_{t-1} g_{t} $   & 0 & 0 & 0 & 0 & 0 & 0 & 1 & 0 & 0 & 0 & 0 & 0 & 0 & 0 & 0 & 0 & 0 \\
	$ \eta_{t-1}^4 $    & 0 & 0 & 0 & 0 & 0 & 0 & 0 & 0 & 0 & 0 & 0 & 0 & 0 & 0 & 0 & 0 & 0 \\
	$ \eta_{t-2}^4 $    & 0 & 0 & 0 & 0 & 0 & 0 & 0 & 0 & 1 & 0 & 0 & 0 & 0 & 0 & 0 & 0 & 0 \\
	$ \eta_{t-3}^4 $    & 0 & 0 & 0 & 0 & 0 & 0 & 0 & 0 & 0 & 1 & 0 & 0 & 0 & 0 & 0 & 0 & 0 \\
	$ \eta_{t-4}^4 $    & 0 & 0 & 0 & 0 & 0 & 0 & 0 & 0 & 0 & 0 & 1 & 0 & 0 & 0 & 0 & 0 & 0 \\
	$ \eta_{t-1}^3 $    & 0 & 0 & 0 & 0 & 0 & 0 & 0 & 0 & 0 & 0 & 0 & 0 & 0 & 0 & 0 & 0 & 0 \\
	$ \eta_{t-2}^3 $    & 0 & 0 & 0 & 0 & 0 & 0 & 0 & 0 & 0 & 0 & 0 & 0 & 1 & 0 & 0 & 0 & 0 \\
	$ \eta_{t-3}^3 $    & 0 & 0 & 0 & 0 & 0 & 0 & 0 & 0 & 0 & 0 & 0 & 0 & 0 & 1 & 0 & 0 & 0 \\
	$ \eta_{t-1}^2 $    & 0 & 0 & 0 & 0 & 0 & 0 & 0 & 0 & 0 & 0 & 0 & 0 & 0 & 0 & 0 & 0 & 0 \\
	$ \eta_{t-2}^2 $    & 0 & 0 & 0 & 0 & 0 & 0 & 0 & 0 & 0 & 0 & 0 & 0 & 0 & 0 & 0 & 1 & 0 \\
	$ \eta_{t-1}^1 $    & 0 & 0 & 0 & 0 & 0 & 0 & 0 & 0 & 0 & 0 & 0 & 0 & 0 & 0 & 0 & 0 & 0 \\
\end{tabular}	\\


$ \Psi $ = 
\begin{tabular}{rrrrr}
	$ \sigma^2_{\epsilon} $ & 0 & 0 & 0 & 0 \\
    0                      & 0                      & 0 & 0 & 0 \\
	0                      & 0                      & 0 & 0 & 0 \\
	0                      & 0                      & 0 & 0 & 0 \\	
	0                      & 0                      & 0 & 0 & 0 \\	
	0                      & 0                      & 0 & 0 & 0 \\
	0                      & 0                      & 0 & 0 & 0 \\		
	0                      & $ \sigma^2_{\eta^1} $  & 0 & 0 & 0 \\
	0                      & 0                      & 0 & 0 & 0 \\
	0                      & 0                      & 0 & 0 & 0 \\
	0                      & 0                      & 0 & 0 & 0 \\
	0                      & 0                      & $ \sigma^2_{\eta^2} $ & 0 & 0 \\
	0                      & 0                      & 0 & 0 & 0 \\		
	0                      & 0                      & 0 & 0 & 0 \\
	0                      & 0                      & 0 & $ \sigma^2_{\eta^3} $ & 0 \\
	0                      & 0                      & 0 & 0 & 0 \\
    0                      & 0                      & 0 & 0 & $ \sigma^2_{\eta^4} $ \\		 										
\end{tabular}	

$ \Pi $ = 
\begin{tabular}{rrr}
	0 & 0 & 0 \\
	0 & 0 & 0 \\	
	0 & 0 & 0 \\	
	0 & 0 & 0 \\	
	1 & 0 & 0 \\	
	0 & 1 & 0 \\	
	0 & 0 & 1 \\
	0 & 0 & 0 \\
	0 & 0 & 0 \\
	0 & 0 & 0 \\
	0 & 0 & 0 \\
	0 & 0 & 0 \\
	0 & 0 & 0 \\
	0 & 0 & 0 \\
	0 & 0 & 0 \\
	0 & 0 & 0 \\
	0 & 0 & 0 \\
\end{tabular}
The impulse response functions for this system are as follows:
\begin{figure}[H]
	\centering
	\includegraphics[scale=.5]{impulse_response}
	\caption{Standard deviation of all shocks = 1}
\end{figure}

\begin{figure}[H]
	\centering
	\includegraphics[scale=.5]{impulse1_10}
	\caption{Standard deviation first news shock = 10, the rest 1}
\end{figure}
\begin{figure}[H]
	\centering
	\includegraphics[scale=.5]{impulse10}
	\caption{Standard deviation of all news shocks = 10}
\end{figure}

From looking at the IRF's with different standard deviations it does not appear like much changes with a change in the size of the news shock. Interest rates and inflation had the greatest response, while output and government spending changed in scale only. 

\section*{Problem 2}
Uncertainty shocks effects on Canadian economy are shown below in the IRF's. 
\begin{figure}[H]
	\centering
	\includegraphics[scale=.5]{irfsCanada}
\end{figure}

Global uncertainty seems to play a small role in the economy, as revealed by the IRF's. It affects all variables negatively, but very small. The uncertainty would have to be rather large for the effect to be measurable in the economy. The shock above is a 1 unit shock in the uncertainty index. If the index increased by a factor of 10 or so then the impulse responses produce some sizeable shocks on the economy. Nevertheless the effect of uncertainty portends bad things for the economy, above all variables show contractionary response to higher uncertainty.

\section*{Problem 3}
Paper refereed: \textit{Measuring Economic Policy Uncertainty}, Scott R. Baker, Nicholas Bloom, Steven J. Davis. \url{http://cep.lse.ac.uk/pubs/download/dp1379.pdf} \\

Decision: Accept with Revisions. \\

I certainly agree with the authors intentions. Measuring economic uncertainty as well as possible is important for the uncertainty literature. To get accurate results from VAR's about uncertainty it must be measured correctly, a problem pervasive in this literature because it is by its very nature a cognitive phenomenon and difficult to quantify. Yet uncertainty about the economy has definitely played a role in creating crises when it is high and palliating them when it is low. The paper cites that uncertainty about policy was responsible in part for the steep decline in 2008-2009 in economic output. \\

The literature about this topic has continued to grow since the most recent crisis, and this paper intends to improve existing indexes on economic uncertainty. There will never be a perfect index, since that is just infeasible, however, a good index must take into account the most reasonable and logical indicators of economic uncertainty that are possibly available. The index proposed by the authors was a word search model that took several key words that indicated uncertainty and searched for them in a series of newspapers. There were some criteria that must be met that indicated the connection of the words truly did indicate uncertainty. The computer results were checked with human created indexes and the results showed high correlation suggesting that the computer version is doing is job approximating what normally a human would have to do quite well. \\ 

On the critical side, how can this paper be improved? The index they have created is important, in addition to a well done monthly index they also included a daily index from a database called Newsbank which is able to collect a wide range of articles daily and get enough hits to construct a daily uncertainty index. However, I wonder if searching newspapers is really the only way to construct an index of this kind. Uncertainties yardstick is not limited to newspaper articles and there was no mention of any other way to gather information about uncertainty other than newspaper articles. Surely other methods of measuring this important variable can be, or have been devised. Then why have the results not been amalgamated to create an even better index? This should be considered. \\

Overall, I believe the paper is very important, and it should be published. The database is a potential goldmine for those in this area of research. It was well done and checked with an admirable contingent of University of Chicago students trained to search for uncertainty in the news. The computer simulations showed that the index matched up well and the importance for the economy is measurable in the usual VAR's. Publishing this paper will definitely be of use to those who are actively involved in this part of economic research. 


\end{document}
