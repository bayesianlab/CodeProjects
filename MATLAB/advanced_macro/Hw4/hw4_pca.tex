\documentclass[]{article}

\usepackage{mathtools}
\usepackage{amsfonts}
\usepackage{amssymb}
\usepackage{fullpage}
\usepackage{amsmath}
\usepackage{multirow}
\usepackage{graphicx}
\usepackage{caption}
\usepackage{subcaption}
\usepackage{float}
\usepackage{hyperref}
\hypersetup{
	colorlinks,
	citecolor=black,
	filecolor=black,
	linkcolor=black,
	urlcolor=black
}
\usepackage[framed,numbered,autolinebreaks,useliterate]{mcode}
\usepackage{listings}
\usepackage{color} %red, green, blue, yellow, cyan, magenta, black, white
\definecolor{mygreen}{RGB}{28,172,0} % color values Red, Green, Blue
\definecolor{mylilas}{RGB}{170,55,241}

%opening
\title{}
\author{}
\date{}
\begin{document}

\maketitle


\begin{table}[H]
	\scalebox{.8}{
\begin{tabular}{rrrrrrrrrrrr}
	& SVENY01 &	SVENY02	& SVENY03	& SVENY04&	SVENY05 &	SVENY06 &	SVENY07 &	SVENY08	& SVENY09 &	SVENY10 \\
	\hline\hline 
PC1 &	0.9549 & 0.9824 & 0.9938 & 0.9983 & 0.9992 & 0.9979 & 0.9954 & 0.9919 & 0.9881 & 0.9840 \\
PC2&	0.0431 & 0.0174 & 0.0054 & 0.0008 & 0.0001 & 0.0017 & 0.0045 & 0.0079 & 0.0115 & 0.0149 \\
	PC3 & 0.0019 & 0.0000 & 0.0006 & 0.0008 & 0.0006 & 0.0002 & 0.0000 & 0.0001 & 0.0004 & 0.0009 \\
\end{tabular}
}
\caption{For each maturity of the zero-coupon yield up to 10-year maturity the table above shows the amount of variance each principal component explains}
\end{table}	

The average for each principal component above gives PC1 = 0.9886, PC2 = 0.0107 and PC3 = 0.0006.	
	
	
	
	
	
	
	



\end{document}
