\documentclass[]{article}

\usepackage{mathtools}
\usepackage{amsfonts}
\usepackage{amssymb}
\usepackage{fullpage}
\usepackage{amsmath}
\usepackage{multirow}
\usepackage{graphicx}
\usepackage{caption}
\usepackage{subcaption}
\usepackage{float}
\usepackage{hyperref}
\hypersetup{
	colorlinks,
	citecolor=black,
	filecolor=black,
	linkcolor=black,
	urlcolor=black
}
\usepackage[framed,numbered,autolinebreaks,useliterate]{mcode}
\usepackage{listings}
\usepackage{color} %red, green, blue, yellow, cyan, magenta, black, white
\definecolor{mygreen}{RGB}{28,172,0} % color values Red, Green, Blue
\definecolor{mylilas}{RGB}{170,55,241}

%opening
\title{Homework 1 Solutions}
\author{Dillon Flannery}

\begin{document}

\maketitle

\section{Results} According to results previously derived analytically the order of accuracy for local truncation error is $ O(h^4) $. Therefore, if we halve the stepsize $ h $ we should expect that the error will be 16 times as small times some constant. By the usual method of dividing the errors from the previous iteration $ O(h^4) $ implies
\[
E_1 = Ch^4
\]
\[
E_2 = C(\frac{h}{2})^4
\]
Then the $ \log_2(E_1/E_2) = 4 \log_2(2) = 4$. Doing this routine in MATLAB produced results for the given differential equation exactly according to theory. This was tested in the following way:
\begin{enumerate}
	\item The analytical solution was calculated, that solution is shown on the handwritten pages.
	\item The approximate solution was created using the 4th-order-Runge-Kutta method.
	\item The true values were calculated for the same time steps as the approximate.
	\item The maximum difference between the two was taken.
	\item This was repeated for many different $ h $'s each decreasing in size by $ 1/2 $.
	\item The $ \log_2  $ of the divided errors was taken.
\end{enumerate}
This produced the following table:
\begin{table}[H]
	\centering
	\caption{Fourth-order Accuracy of Runge-Kutta Method}
	\begin{tabular}{rrr}
		$ h_i $ & $ h_{i+1} $ & $ \log_2(E_i/E_{i+1}) $\\
		\hline 
		\hline 
		1 & 0.5 & 3.93\\
		0.5 & 0.25 & 4.0875\\
		0.25 & 0.125 & 4.0937\\
		0.125 & 0.0625 & 4.0650\\ 
		0.0625 &  0.03125 & 4.0503 \\
		0.03125 & 0.015625 &  4.2405\\
		\hline
	\end{tabular}
\end{table}
	
\subsection*{Nonlinear ODE of 2 Unknowns} I was able to construct a nonlinear ODE of two unknowns that had periodic solutions. The order of accuracy was less than fourth order, and the accuracy decayed as the time span increased. The ability of the Runge-Kutta method to approximate the correct answers depended on its nearness to the initial conditions. It diverged far off the path if the step was too large and the solutions were sufficiently far from the initial conditions. It was approximately fourth order, but it had obvious error as it increased in $ t $ time. 

The system I constructed was a well known Van Der Pol equation, 
\[ y'' - (1 - y^2) y' + y = 0  \]
By a substitution of variables, 
\[ y' = x \]
\[ x' = (1-y^2)x - y \]
And initial conditions $ [2,0] $, the solution is clearly periodic, 
\begin{figure}\centering
	\caption{Solution of differential equation system, $ t = [0,20] $}
\includegraphics[scale=.5]{vanderpol}
\end{figure}
The period here appears to be around 12.

I  tested the accuracy of my Runge-Kutta method by comparing it to MATLAB's \mcode{ode113} function. This function is MATLAB's best function for high accuracy solutions to ODE's. I set the \mcode{odeset} options to have a relative tolerance of $ 10^-9 $, well beyond four order. Then using this as the true values I performed the same routine as above. The results are shown below, 

\begin{table}[H]
	\centering
	\begin{tabular}{rrr}
		$ h_i $ & $ h_{i+1} $ & $ \log_2(E_i/E_{i+1}) $\\
		\hline 
		\hline 
		0.5 & 0.25 & 3.5873\\
		0.25 & 0.125 & 3.7324\\
		0.125 & 0.0625 & 3.8194\\ 
		0.0625 &  0.03125 & 3.1764 \\
		0.03125 & 0.015625 &  1.0444\\
	\end{tabular}
\end{table}
	
\appendix
\section{Code}	
	\listinputlistings{runge4.m}
	
	
	
	
	
	



\end{document}
