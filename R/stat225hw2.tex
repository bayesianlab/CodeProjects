\documentclass{article}
\usepackage{graphicx}
\usepackage{hyperref}
\usepackage{amsmath}
\usepackage{times}
\usepackage{float}
\usepackage{hyperref}

\usepackage{Sweave}
\begin{document}
\Sconcordance{concordance:stat225hw2.tex:stat225hw2.Rnw:%
1 5 1 1 0 72 1}

\section*{Question 4}
Poisson Likelihood gives a Jeffreys Prior by $ (I(\theta))^{\frac{1}{2}} $
$$
I(\theta) =  V [ \frac{\partial \log f(x|\theta)}{\partial \theta} ]
$$
$$
V \Big[\frac{\partial \log f(x|\theta)}{\partial \theta}\Big] = \frac{\sum_i^N x_i}{\theta} - N =  \frac{\sum_i^N V(x_i)}{\theta^2} = \frac{N}{\theta} 
$$
Jeffreys Prior ($\pi_J(\theta)) $ is 
$$
\Big(\frac{N}{\theta}\Big)^{\frac{1}{2}}
$$
The posterior will be a Gamma $(Ga(\alpha_n, \beta_n))$ $\alpha_n = \frac{1}{2} + \sum_{i=1}^N x_i $ and $\beta_n = 1/N $
$$
\pi(\theta|x) \propto f(x|\theta)\times \pi(\theta) =  \theta^{\sum x_i} e^{N\theta} \times \theta^{-\frac{1}{2}}
$$
$$
\pi(\theta|x) \propto \theta^{\alpha_n} e^{(\theta/\beta_n)}
$$
Where $\alpha_n = \frac{1}{2} + \sum_{i=1}^N x_i $ and $\beta_n = 1/N $.
\\
In this question, $ \sum_{i=1}^N x_i = 196 $  and $ N = 280 $. The Poisson mean is $ 0.70 $. 

\subsection*{Relative Error Loss i)}
Minimize expected posterior loss:
\begin{align*}
\min E_{\theta|X}[L(\theta, a)] = & E \frac{\partial}{\partial a} L(\theta, a) = 0\\ 
 E[ (\frac{a}{\theta^2} - \frac{1}{\theta})] = & \ 0 \\
 a E[\frac{1}{\theta^2}] = & E[\frac{1}{\theta}] \\ 
 a^* = \frac{E \Big[\frac{1}{\theta^2}\Big]  }{E\Big[ \frac{1}{\theta} \Big]}
\end{align*}
The integration was performed in R, 
\begin{Schunk}
\begin{Sinput}
> parti <- function(theta,k){
+   res <- theta^(-k) * dgamma(theta, 196.5, rate =280)
+ }
> num <-integrate(parti, 0, Inf, k=1)
> den <- integrate(parti,0,Inf,k=2)
> cat("astar = ", num[[1]]/den[[1]])
\end{Sinput}
\begin{Soutput}
astar =  0.6946429
\end{Soutput}
\end{Schunk}
Which is very close to the Poisson mean, 0.70.
\subsection*{Squared Error Loss ii)}
By Robert Proposition 2.5.1 with squared error loss the Bayes estimator is the posterior expectation,
$$
a^* = E[\theta|X]
$$
Since the  distribution is $Ga(\alpha_n, \beta_n)$ the posterior mean is $\frac{196.5}{280} \approx 0.7018$, no need to use R for this.
\subsection*{Entopy Loss iii)}

\begin{align*}
\min E_{\theta|X} [L(\theta, a)] = & E \frac{\partial}{\partial a} L(\theta, a) = 1 -E \frac{\theta}{a} \\
a^*  =& E[\theta|X] 
\end{align*}
\\
Since the  distribution is $Ga(\alpha_n, \beta_n)$ the posterior mean is $\frac{196.5}{280} \approx 0.7018$

\section*{Part b)}
For the following let $ \phi(\theta) = \frac{r_0}{\theta + r_0} $ %and $ \frac{\partial }{\partial \theta} \phi(\theta) = -\frac{r_0}{ (\theta + r_0)^2 } $
\subsection*{b.1)}

\begin{align*} 
I(\theta) = &  V(\frac{\partial \log(f(x|\phi)}{\partial \phi}) = V \Big[\frac{x}{\phi} - \frac{r_0}{1-\phi} \Big] \\
I(\theta) = & \frac{r_0}{\phi(1-\phi)}
\end{align*}
The Jeffrey's Prior will be $ \sqrt{I(\phi)} = \sqrt{\frac{r_0}{\phi(1-\phi)^2}} $

\subsection*{b.2)} Find the posterior distribution with this prior. The posterior distribution is a Beta distribution, $ Be(\alpha_n, \beta_n) $ with $\alpha_n = n\bar{x} + \frac{1}{2} $ and $\beta_n = nr_0 $,
$$ f(x;\phi) ={ {x + r_0 - 1 }\choose{x}}^N  \phi^{n \bar{x} } (1-\phi)^{n r_0}  $$
$$ \pi(\phi) = \sqrt{I(\phi)} $$
The posterior is then:
\begin{align*}
\pi(\phi|x) \propto & \phi^{\sum x- \frac{1}{2}} (1 - \phi)^{n r_0 - 1} \\
\pi(\phi|x) \propto &\phi^{\alpha_n - 1}(1 - \phi)^{\beta_n - 1}
\end{align*}
This is the kernel of a Beta distribtution with the parameters given above. 
\subsubsection*{b.2.i}
With R one gets:
\begin{Schunk}
\begin{Sinput}
> partb <- function(theta, k){
+   res <- theta^(-k) * dbeta(theta, 196.5, 1120)
+ }
> num <- integrate(partb, 0, Inf, k=1)
> den <- integrate(partb, 0, Inf, k=2)
> cat(num[[1]]/den[[1]])
\end{Sinput}
\begin{Soutput}
0.147965
\end{Soutput}
\end{Schunk}
\subsubsection*{b.2.ii-iii}
The posterior mean is 
$$ 
E [\phi|x] = \frac{\alpha_n}{\alpha_n + \beta_n} = \frac{196.5}{1120 + 196.5} \approx 0.15
$$
\newpage

\section*{Question 3}
\begin{Schunk}
\begin{Sinput}
> Y <- rnorm(10000,0,5)
> eY <- exp(Y)
> X <- eY/(1+eY)
> plot(density(X), main="Density of X")
\end{Sinput}
\end{Schunk}
\includegraphics{stat225hw2-q3}
\begin{Schunk}
\begin{Sinput}
> plot(hist(X))
\end{Sinput}
\end{Schunk}
\includegraphics{stat225hw2-q3a}
Show (using R) the two densities are the same:
\begin{enumerate}
\item Code for partition:
\begin{Schunk}
\begin{Sinput}
> weightedBeta <- function(X,cuts){
+   Fn <- ecdf(X)
+   sto <- Fn(1/cuts)*dbeta(X,1,cuts)
+   for(j in 2:cuts){
+     jm1 <- j-1
+     b <- j/cuts
+     a <- jm1/cuts
+     beta <- cuts - j + 1
+     sto <- sto + ((Fn(b) - Fn(a))*dbeta(X,j,beta))
+   }
+   sto
+ }
\end{Sinput}
\end{Schunk}
\item Densities are the same as the number of partitions increase,
\begin{figure}[H]
\centering
\begin{Schunk}
\begin{Sinput}
> par(mfrow=c(1,4))
> Z <- weightedBeta(X, 3)
> plot(density(Z), main="Density Z")
> Z <- weightedBeta(X, 6)
> plot(density(Z), main="Density Z")
> Z <- weightedBeta(X, 18)
> plot(density(Z), main="Density Z")
> Z <- weightedBeta(X, 24)
> plot(density(Z), main="Density Z")
\end{Sinput}
\end{Schunk}
\includegraphics{stat225hw2-006}
\caption{Partions = 3, 6, 18, 24}
\end{figure}
\end{enumerate}
\newpage 

\section*{Question 6}
We are to test multiple hypothesis for the paramter $ \mu_i $. If it is significantly different than zero then I will reject the null hypothesis $ H_0 $, otherwise I will fail to reject. The code to run the Rjags for this question is here:
\begin{Schunk}
\begin{Sinput}
> library(data.table)
> library(rjags)
> X <-fread('http://www.ics.uci.edu/~mguindan/teaching/stats225/zscores.txt')
> p6data <-  list(y = X$V1, N = length(X$V1))
> mod <- jags.model("~/Google Drive/CodeProjects/R/p6.bug", data = p6data, n.chains=5, n.adapt = 100)
\end{Sinput}
\begin{Soutput}
Compiling model graph
   Resolving undeclared variables
   Allocating nodes
Graph information:
   Observed stochastic nodes: 200
   Unobserved stochastic nodes: 402
   Total graph size: 1608

Initializing model
\end{Soutput}
\begin{Sinput}
> res <- coda.samples(mod, var = c("y", "mu"), n.iter=1000,thin =10)
> summ <- summary(res)
> R <- data.frame(as.matrix(res))
> sto <- numeric(200)
> for(i in 1:200){
+   sto[i]<-quantile(R[,i],.5)
+ }
\end{Sinput}
\end{Schunk}
And the Jags code itself is here:
\begin{verbatim}
model{
p0 ~ dbeta(1,1)
sigma2 ~ dgamma(.001,.001)
for(i in 1:N){
  y[i] ~ dnorm(mu[i]*gamma[i], 1/(sigma2
  ))
  mu[i] ~ dnorm(0, 1)
  gamma[i] ~ dbern(p0)
}
}
\end{verbatim}

My decision rule needs to be based upon the median model. If the 50\% quantile is significantly far from 0 then this would indicate evidence against the null. With enough such evidence I will reject $H_0$. A plot of the 50\% of $ \mu_i $ is show here:
\begin{figure}[H]
\centering
\begin{Schunk}
\begin{Sinput}
> plot(sto, main="50% Percentiles of mean's", ylab ="Percentile", col="blue")
\end{Sinput}
\end{Schunk}
\includegraphics{stat225hw2-008}
\end{figure}
For this question I would consider horizontal lines to define the cutoff for the $ \mu_i $'s, say $ \pm  k $, perhaps $ k = \frac{1}{2} $. For those falling above this threshold I would reject the null in favor of the alternative. 

\end{document}
